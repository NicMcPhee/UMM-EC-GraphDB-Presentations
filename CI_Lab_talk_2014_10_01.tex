\documentclass{beamer}

\mode<presentation>
{
  \usetheme{CambridgeUS}
  \setbeamercovered{transparent}
}

\usepackage[english]{babel}
\usepackage[latin1]{inputenc}
\usepackage{times}
\usepackage[T1]{fontenc} 
\usepackage{amsmath}

\newcommand{\linespace}{\vskip 0.25cm}

% The text in square brackets is the short version of your title and will be used in the
% header/footer depending on your theme.
\title[Analysis of GP Ancestry in Neo4j]{Analysis of Ancestry in Genetic Programming with a Graph Database}

% Sub-titles are optional - uncomment and edit the next line if you want one.
% \subtitle{Why does sub-tree crossover work?} 

% The text in square brackets is the short version of your name(s) and will be used in the
% header/footer depending on your theme.
\author[McPhee]{Nicholas Freitag McPhee \\ with help from \\ David Donatucci and M. Kirbie Dramdahl}

% The text in square brackets is the short version of your institution and will be used in the
% header/footer depending on your theme.
\institute[UMM]
{
  Division of Science and Mathematics \\
  University of Minnesota, Morris \\
  Morris, Minnesota, USA
}

% The text in square brackets is the short version of the date if you need that.
\date[Oct '14, Hamp. CI-Lab] % (optional)
{1 Oct 2014 \\ Hampshire College CI-Lab}

% Delete this, if you do not want the table of contents to pop up at
% the beginning of each subsection:
\AtBeginSection[]
{
  \begin{frame}<beamer>
    \frametitle{Outline}
    \tableofcontents[currentsection, hideothersubsections]
  \end{frame}
}

\begin{document}

\begin{frame}
  \titlepage
\end{frame}

% For a 20-25 minute senior seminar talk you probably want something like:
% - Two or three major sections (other than the summary).
% - At *most* three subsections per section.
% - Talk about 30s to 2min per frame. So there should probably be between
%   15 and 30 frames, all told.

\section*{Overview}

\subsection*{The Big Picture}

\begin{frame}
  \frametitle{The Big Picture}
  
  \begin{itemize}
	\item Genetic programming (GP) demonstrably works.
	\item Difficult to determine how this process works.
	\item Databases allow examination of the internal interactions of a run.
	\item Graph databases more efficient for this task than relational DBs.
	\item Analysis may allow us to improve GP tools.
  \end{itemize}
\end{frame}

\subsection*{Outline}

\begin{frame}
  \frametitle{Outline}
  \tableofcontents[hideallsubsections]
\end{frame}

\section[Genetic Programming]{Genetic Programming}

\subsection{Transformations and genealogical history}

\begin{frame}
\frametitle{Transformations and genealogical history}
\begin{columns}
\begin{column}{0.55\textwidth}
{\footnotesize
		\begin{description}
		\item[Crossover] sexual reproduction \\ (\textcolor{blue}{root} and \textcolor{red}{non-root})
		\item[Mutation] subtrees altered
		\item[Reproduction] asexual reproduction
		\item[Elitism] reproduction based on fitness
		\end{description}
}
\end{column}
\begin{column}{0.45\textwidth}
{\tiny \hspace{0.3cm} \textcolor{blue}{Root} \hspace{1cm} \textcolor{red}{Non-Root} \hspace{1cm} Child}
\includegraphics[width=.95\textwidth]{subtreeCrossover.jpg} \\
{\tiny \hspace{2.5cm} \url{geneticprogramming.us}}
\end{column}
\end{columns}
\vspace*{-0.5cm}
\begin{center}
\includegraphics[width=.9\textwidth]{XO_path_example.pdf}
\end{center}
\end{frame}

\section[Graph DB]{Graph Database}
\subsection{Neo4j}

\begin{frame}
	\frametitle{Neo4j}
	
	
	
\begin{columns} 
\begin{column}{0.60 \textwidth}
Neo4j is a graph database.
		\begin{itemize}
		\item Relatively new tool
			\begin{itemize}
			\item Initial release 2007
			\item V1.0 in 2010, V2.0 last December
			\end{itemize}
		\item Information is stored using a graph consisting of nodes and relationships
		\item Efficient recursive queries compared with traditional databases
		\end{itemize}
		\end{column}
		\begin{column}{0.40\textwidth}
   \includegraphics[width=0.95\textwidth]{graphdb-neo4j.png}
       \\
    \only{\tiny{Neo4j \ \url{http://goo.gl/nzRWSV}}}
  \end{column}
  \end{columns}

\end{frame}

\subsection{Cypher}

\begin{frame}
	\frametitle{Cypher}
	Neo4j's query language is Cypher.
	\begin{columns}
	\begin{column}{0.40\textwidth}

	Fundamental elements of Cypher queries:
		\begin{itemize}
		\item START
		\item MATCH
		\item WHERE
		\item RETURN
		\end{itemize}
	\end{column}
	\begin{column}{0.60\textwidth}

	\includegraphics[width=.95\textwidth]{parents.pdf}
	\linebreak
	\emph{
START parent=node(43)
\linebreak
MATCH (parent)-[:PARENTOF]->(child)
\linebreak
\textbf{WHERE id(child) < 47}
\linebreak
RETURN parent, child;
}

	\end{column}	
	\end{columns}
\end{frame}

\subsection{DB setup}

\begin{frame}
\frametitle{DB setup}

We store:
\begin{itemize}
	\item Individuals as nodes + data, e..g, program, fitness, generation
	\item XO, mutation, replication events as edges between nodes
\end{itemize}

~

Different edge types indicate relationship types, e.g., \textcolor{blue}{\emph{ROOT\_PARENT}} edge from 991 to 1000 says 991 is the root-parent of 1000.

\begin{center}
\includegraphics[width=.9\textwidth]{XO_path_example.pdf}
\end{center}

\end{frame}

\subsection{Cypher examples}

\begin{frame}
\frametitle{Finding the ``winning'' fitness and individual}

Finding the minimal fitness across all individuals (nodes) in the DB:

~ 

\texttt{start n=node(*) \\
return min(n.penalizedFitness);
}

~

Finding one (arbitrary) ``winning'' individual:

~

\texttt{start n=node(*) \\
return n \\
order by n.penalizedFitness \\
limit 1;
}

~

These are not unlike queries in relational databases.

\end{frame}

\begin{frame}
\frametitle{Ancestry of an individual (hard for relational DBs)}

Find all the root ancestors of individual 99,000:

~

\texttt{start n=node(99000) \\
match ps = (n)<-[r:ELITISM|PARENTOF|ROOT\_XOOF|MUTANTOF*]-(p) \\
return distinct ps;
}

~

Find the fitness of all the root ancestors of individual 99,000:

~

\texttt{start n=node(99000) \\
match ps = (n)<-[:ELITISM|PARENTOF|ROOT\_XOOF|MUTANTOF*]-(p) where p.generation=1 \\
return extract(x in nodes(ps) | x.fitness);
}

\end{frame}

\section[Results]{Results}

\subsection{A few questions}

\begin{frame}
\frametitle{A few questions}
\begin{enumerate}
\item What does the fitness of the ``winning'' individual's ancestry line look like over time?
\item How often does mutation improve fitness? Also, how often does crossover improve fitness, where the root parent is more fit than the non-root parent, and vice versa?
\item Does a group of individuals have a common root parent ancestor and what is the latest generation where such an ancestor occurs?
\item How many individuals in the initial generation have any root parent descendants in the final generation?
\end{enumerate}
\end{frame}

\subsection[Fitness Graph]{Fitness Over Time}

\begin{frame}
\frametitle{Fitness Over Time}
\emph{What does the fitness of the ``winning'' individual's ancestry line look like over time?}
\begin{center}
\includegraphics[width=0.85\textwidth]{Combined_fitness_over_time_no_dashed}
\end{center}
\end{frame}

\subsection[Improved Transformations]{Improved Transformations}

\begin{frame}
\frametitle{Percentage of Improved Transformations}
\emph{How often do mutation and crossover improve fitness?}
\begin{center}
{\tiny Results for one 10,000 individual run and three 1,000 individual runs}
\includegraphics[width=0.95\textwidth]{All_percentages_over_time}
\end{center}
\end{frame}

\subsection{Common Ancestor(s)}

\begin{frame}
\frametitle{Common Ancestor}
\emph{How far back to a common root parent ancestor? How many initial generation individuals have descendants in the final generation?}
\begin{columns}
\begin{column}{0.55\textwidth}
In run w/ 10K individuals for 100 gens:
\begin{itemize}
	\item Final pop consists of 2 clades
	\item One clade \emph{strongly} dominates (only 24 left in small clade)
	\item Each clade root-descends from single individual in initial population
\end{itemize}
\end{column}
\begin{column}{0.45\textwidth}
\begin{center}
\includegraphics[height=0.70\textheight]{subset_confluence_trimmed}
\end{center}
\end{column}
\end{columns}
\end{frame}

\section{Conclusions and future work}

\begin{frame}
\frametitle{Conclusions}

\begin{itemize}
\item We can collect \& extract useful data about evolutionary dynamics
\item More detailed analysis that moves beyond statistical summaries
\item Suggests interesting avenues to explore
	\begin{itemize}
		\item Importance of having root parent be more fit
		\item Changing role of mutation over the life of a run
		\item Potential for clades/quasi-species and their impact
	\end{itemize}
\end{itemize}

~

Concerns about scaling
\begin{itemize}
	\item Can we do lots of runs in parallel pouring data into a central DB?
	\item Can runs be distributed across the network and still perform well?
	\item How much memory is needed to work with large datasets?
\end{itemize}

\end{frame}

\begin{frame}
\frametitle{Future work}

Requiring the root parent to have better fitness in XO
\begin{itemize}
	\item Explore its impact across a variety of problems
	\item Explore equivalents in other EC systems
	\begin{itemize}
		\item What's the ``equivalent'' (if any) in Push?
		\item Ensure parent that contributes the tail is more fit?
	\end{itemize}
\end{itemize}

~

Dynamically set parameters
\begin{itemize}
	\item This has been done a lot, but usually in an ad hoc way.
	\item Can we use data collected from runs to do this in a more principled way?
\end{itemize}
\end{frame}

\begin{frame}
	\frametitle{Thanks!}
	
	Thank you for your time and attention!

~
	
	Contact:
	\begin{itemize}
		\item \texttt{mcphee@morris.umn.edu}
		\item Twitter @NicMcPhee
		\item Paper: \url{https://github.com/NicMcPhee/MICS-2014-GraphDB-EC}
	\end{itemize}

~

Big thanks to David Donatucci and Kirbie Dramdahl for their many contributions to this work.
	
	\begin{center}
	{\huge Questions?}
	\end{center}
\end{frame}

\section*{References}

\begin{frame} 
\frametitle{References}
\nocite{*}
\bibliographystyle{acm}
{\tiny \bibliography{MICS_2014}}
\end{frame} 

% \section[Setup]{Experimental Setup}

% \subsection{Configurations}

\begin{frame}
\frametitle{Run Configurations}
\begin{description}[align=left, leftmargin=*]
{\small
\item[Target Function] sin(x)
\item[Variables] x (range 0.0 to 6.2, incremented by steps of 0.1)
\item[Constants] range between -5.0 and 5.0
\item[Operations] addition (+), subtraction (-), multiplication (*), protected division (/)
\item[Generation Number] 100
\item[Population Size Per Gen] 1,000 (3 runs) and 10,000 (1 run)
\item[Transform Percentages] crossover (90\%), mutation (1\%), reproduction (9\%)
\item[Elitism] best 1\%
\item[Fitness] absolute error between target function and individual function
}
\end{description}
\end{frame}

\end{document}
